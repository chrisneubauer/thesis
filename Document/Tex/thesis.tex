%%%%%%%%%%%%%%%%%%%%%%%%%%%%%%%%%%%%%%%%%%%%%%%%%%%%%%%%%%%%%%%%%%%%%%%%%%%%%%%
% Frame for Bachelor or Master Thesis at FAU/i1
% Topic: [Title of thesis]
% Copyright (C) 2012  [Author]
% based on original work by Johannes Goetzfried, used with permission
%%%%%%%%%%%%%%%%%%%%%%%%%%%%%%%%%%%%%%%%%%%%%%%%%%%%%%%%%%%%%%%%%%%%%%%%%%%%%%%
%
% usage hints:
% - base document class is scrbook with some fancy additions
% - can be used for English or German texts (Umlaut encodings enabled)
% - inserts hyperlinks for cross referencing
% 
% included files contain examples for
% - listings
% - tables
% - figures
%
% Use at your own risk, i1 offers only limited support for this frame.
%
%%%%%%%%%%%%%%%%%%%%%%%%%%%%%%%%%%%%%%%%%%%%%%%%%%%%%%%%%%%%%%%%%%%%%%%%%%%%%%%%

% Include Header
%%%%%%%%%%%%%%%%%%%%%%%%%%%%%%%%%%%%%%%%%%%%%%%%%%%%%%%%%%%%%%%%%%%%%%%%%%%%%%%
%
% Header
% 
%%%%%%%%%%%%%%%%%%%%%%%%%%%%%%%%%%%%%%%%%%%%%%%%%%%%%%%%%%%%%%%%%%%%%%%%%%%%%%%

% Font size and paper
\documentclass[10pt,twoside,a4paper,bibliography=totoc]{scrbook}

% Margin
\usepackage{setspace}
\usepackage{anysize}
\marginsize{3cm}{3cm}{2cm}{2cm}

% Font, Encoding
\usepackage[T1]{fontenc}
\usepackage[utf8]{inputenc}
\usepackage{times}

% Fancy header and footer
\usepackage{fancyhdr}

% Graphics
\usepackage{graphicx}
\usepackage{float}
\graphicspath{{images/}}

% Math stuff
\usepackage{amsmath}
\usepackage{amssymb}
\usepackage{nicefrac}

% Code listings
\usepackage{listings}
\lstset{basicstyle=\footnotesize\ttfamily}

% Links within pdf file
\usepackage{hyperref}
\hypersetup{
	colorlinks=true,
	linkcolor=black,
	citecolor=black,
	filecolor=black,
	urlcolor=black,
	breaklinks=true,
	bookmarksnumbered=true,
	pdfstartpage={1},
        % adapt following lines if you want these items to show in pdf
        % otherwise remove
	pdftitle={Design and Implementation of a fault-tolerant form processing application using machine learning},
	pdfsubject={Master Thesis},
	pdfauthor={Christoph Neubauer}
}

% Figure captions
\usepackage{caption}
\captionsetup{font=small,labelfont=bf}

% for sample text, can be removed in production version
\usepackage{blindtext}

% Fancy toc title
\renewcommand{\contentsname}{CONTENTS}

% Fancy chapter page
\makeatletter
\def\@makechapterhead#1{
  \vspace*{100\p@}
  {\parindent \z@ 
    {\raggedleft
      \fontsize{15ex}{15ex}
      \textsf\thechapter\par\nobreak}
    \par\nobreak
    \interlinepenalty\@M
    {\raggedright \Huge \textsf{\textsc{#1}}}
    \par\nobreak
    \leavevmode \leaders \hrule height 0.65ex \hfill \kern \z@
    \par\nobreak
    \vskip 100\p@
  }
}

% No headers on empty pages before new chapter
\makeatletter
\def\cleardoublepage{\clearpage\if@twoside \ifodd\c@page\else
  \hbox{}
  \thispagestyle{plain}
  \newpage
  \if@twocolumn\hbox{}\newpage\fi\fi\fi}
\makeatother \clearpage{\pagestyle{plain}\cleardoublepage}

% Fancy header / footer for chapter pages
\fancypagestyle{plain}{
	\fancyhf{}
	\renewcommand{\headrulewidth}{0pt}
	\fancyfoot[LE,RO]{\thepage}
}

% Fancy header / footer for normal pages
\pagestyle{fancy}
\fancyhf{}
\fancyfoot[LE,RO]{\thepage}
\fancyhead[LO]{\leftmark}
\fancyhead[RE]{\rightmark}

% No indent for paragraphs
\setlength{\parskip}{1.3ex plus 0.2ex minus 0.2ex}
\setlength{\parindent}{0pt}

% A few commands
\newcommand{\term}[1]{\textit{#1}}
\newcommand{\code}[1]{\texttt{#1}}

% Fancy bibliography, see http://merkel.zoneo.net/Latex/natbib.php
\usepackage[square,comma,numbers,sort&compress]{natbib}


\begin{document}

% Alphabetic page numbers
\pagenumbering{alph}

% Title
%%%%%%%%%%%%%%%%%%%%%%%%%%%%%%%%%%%%%%%%%%%%%%%%%%%%%%%%%%%%%%%%%%%%%%%%%%%%%%%
%
% Title
%
%%%%%%%%%%%%%%%%%%%%%%%%%%%%%%%%%%%%%%%%%%%%%%%%%%%%%%%%%%%%%%%%%%%%%%%%%%%%%%%

\begin{titlepage}

\titlehead{
	\centering
	\begin{tabular}[ht]{lcr}
		\parbox{3cm}{
			\centering
			\includegraphics[width=2.5cm]{fau-logo.png}
		} &
		\parbox{5cm}{
			\centering
			Lehrstuhl für Informatik 5 \\
			Friedrich-Alexander-Universität \\
			Erlangen-Nürnberg \\
		} &
		\parbox{3cm}{
			\centering
			\includegraphics[width=2.5cm]{i5-logo.png}
		}
	\end{tabular}
	\vspace{4em}
}

\subject{
	MASTER THESIS
}

% you should also adapt entries for name and title in 00-0-header.tex
% in the pdf tags (if you want)
\title{
  Design and Implementation of a fault-tolerant form processing application using machine learning
}

\author{
	\vspace{4em}
	Christoph Neubauer
}

\date{
	Erlangen, \today
}

\publishers{
	\begin{tabular}{lcl}
		Examiner:  && Prof. Dr.-Ing. habil. Andreas Maier \\
		Advisor:   && PD Dr.-Ing. habil. Peter Wilke \\
		Second Advisor: && Prof. Dr. Luiz Eduardo S. Oliveira
	\end{tabular}
}

\maketitle
\end{titlepage}


% Roman page numbers
\pagenumbering{roman}

% Pre content
%%%%%%%%%%%%%%%%%%%%%%%%%%%%%%%%%%%%%%%%%%%%%%%%%%%%%%%%%%%%%%%%%%%%%%%%%%%%%%%
%
% Declaration 
%
%%%%%%%%%%%%%%%%%%%%%%%%%%%%%%%%%%%%%%%%%%%%%%%%%%%%%%%%%%%%%%%%%%%%%%%%%%%%%%%


% Pseudo chapter
\chapter*{\ }


\vspace*{\fill}


% Header
\begin{Large}
\chapter{Eidesstattliche Erklärung / Statuatory Declaration}
%	\textbf{Eidesstattliche Erklärung / Statutory Declaration}
\end{Large}
\vspace{1.5em}


\noindent\hrule

% German
Hiermit versichere ich eidesstattlich, dass die vorliegende Arbeit von mir
selbständig, ohne Hilfe Dritter und ausschließlich unter Verwendung der
angegebenen Quellen angefertigt wurde. Alle Stellen, die wörtlich oder
sinngemäß aus den Quellen entnommen sind, habe ich als solche kennt\-lich
gemacht. Die Arbeit wurde bisher in gleicher oder ähnlicher Form keiner anderen
Prüfungsbehörde vorgelegt. 
\vspace{1.5em}


% English
I hereby declare formally that I have developed and written the enclosed thesis
entirely by myself and have not used sources or means without declaration in
the text. Any thoughts or quotations which were inferred from the sources are
marked as such. This thesis was not submitted in the same or a substantially
similar version to any other authority to achieve an academic grading. 

\noindent\hrule

\vspace{2em}

% remove the following text if it doesn't seem appropriate

Der Friedrich-Alexander-Universität, vertreten durch den Lehrstuhl
für Informatik 1, wird für Zwecke der Forschung und Lehre ein
einfaches, kostenloses, zeitlich und örtlich unbeschränktes
Nutzungsrecht an den Arbeitsergebnissen der Arbeit einschließlich
etwaiger Schutz- und Urheberrechte eingeräumt.


\vspace{2em}

% Sign
Erlangen, \today
\begin{flushright}
	\underline{\ \ \ \ \ \ \ \ \ \ \ \ \ \ \ \ \ \ \ \ \ \ \ \ \ 
		\ \ \ \ \ \ \ \ \ \ \ \ \ \ \ \ \ \ \ \ \ \ \ \ \ \ \ \ \ 
	} \\
	\small{[YOUR NAME]}
\end{flushright}

%%%%%%%%%%%%%%%%%%%%%%%%%%%%%%%%%%%%%%%%%%%%%%%%%%%%%%%%%%%%%%%%%%%%%%%%%%%%%%%
%
% Abstract
% 
%%%%%%%%%%%%%%%%%%%%%%%%%%%%%%%%%%%%%%%%%%%%%%%%%%%%%%%%%%%%%%%%%%%%%%%%%%%%%%%

% Pseudo chapter
\chapter*{\ }


\begin{center}
	\begin{large}
		\textbf{Zusammenfassung}
	\end{large}
\end{center}
\vspace{0.75em}

Zusammenfassung auf Deutsch \blindtext

\vspace{2em}
\begin{center}
	\begin{large}
		\textbf{Abstract}
	\end{large}
\end{center}
\vspace{0.75em}

Zusammenfassung auf Englisch \blindtext{}



% Table of Contents
 \listoftables
 \listoffigures
\begin{onehalfspacing}
\tableofcontents
\end{onehalfspacing}
\cleardoublepage

% Arabic page numbers
\pagenumbering{arabic}


% Actual content, can be split up into multiple files or kept in
% one big file
\fancyhead[RE]{\leftmark}
%%%%%%%%%%%%%%%%%%%%%%%%%%%%%%%%%%%%%%%%%%%%%%%%%%%%%%%%%%%%%%%%%%%%%%%%%%%%%%%
%
% Introduction
% 
%%%%%%%%%%%%%%%%%%%%%%%%%%%%%%%%%%%%%%%%%%%%%%%%%%%%%%%%%%%%%%%%%%%%%%%%%%%%%%%


\chapter{Introduction}

Some general information on the context and setting. \blindtext{}

\Blindtext[2][2]

\section{Motivation}

Specific motivation for the problem at hand. \blindtext{}

\blindtext[2]

\section{Task}

Concrete task to be solved. \blindtext{}

\blindtext{}


\section{Related Work}

Other relevant academic work and how it differs from this work, for
example \citet{shannon_diff} and \citet{blowfish}. Distinguish between
``textual'' citation, as shown in \citet{shannon_diff}, and
``parenthesis'' citation \citep{blowfish}.

\Blindtext[3][1]


\section{Results}

What has been achieved in this work? \blindtext{}

\blindtext[2][1]

\section{Outline}

How is the thesis structured and why? \blindtext{}


\section{Acknowledgments}

A big thank you for the support to \ldots \blindtext


\fancyhead[RE]{\rightmark}

%%%%%%%%%%%%%%%%%%%%%%%%%%%%%%%%%%%%%%%%%%%%%%%%%%%%%%%%%%%%%%%%%%%%%%%%%%%%%%%
%
% Background
% 
%%%%%%%%%%%%%%%%%%%%%%%%%%%%%%%%%%%%%%%%%%%%%%%%%%%%%%%%%%%%%%%%%%%%%%%%%%%%%%%


\chapter{Background}
\label{sec:background}

\blindtext[3][2]

\begin{figure}[ht]
	\centering
	\includegraphics[width=0.92\textwidth]{screenshot}
	\caption{Some sample caption}
	\label{fig:cbc_enc}
\end{figure}


In Figure~\ref{fig:cbc_enc} you can see how to refer to figures in text. \blindtext{}




\begin{figure}
\centering
\footnotesize
\begin{minipage}[b]{0.50\textwidth}
\centering
\begin{alignat*}{3}
	\hat B_0     & := &\;& IP(P)           \\
	\hat B_{i+1} & := &&   R_i(\hat B_i)   \\
	C            & := &&   FP(\hat B_{32}) \\
	\text{where~~~~~~~~~~~~~~~~~~~~~}      \\
	R_i(X)       & =  &&   L(\hat S_i(X \oplus \hat K_i))
						&\qquad& i = 0, \ldots, 30 \\
	R_i(X)       & =  &&   \hat S_i(X \oplus \hat K_i) \oplus \hat K_{32}
						&& i = 31
\end{alignat*}
\caption{Left part of a complex figure}
\label{fig:serpentcode}
\end{minipage}
\hspace{0.25cm}
\vline
\hspace{0.25cm}
\begin{minipage}[b]{0.40\textwidth}
\centering
\begin{align*}
	X_0,X_1,X_2,X_3 & := \hat S_i(\hat B_i \oplus \hat K_i) \\
	X_0             & := X_0 <<< 13 \\
	X_2             & := X_2 <<< 3 \\
	X_1             & := X_1 \oplus X_0 \oplus X_2 \\
	X_3             & := X_3 \oplus X_2 \oplus (X_0 << 3) \\
	X_1             & := X_1 <<< 1 \\
	X_3             & := X_3 <<< 7 \\
	X_0             & := X_0 \oplus X_1 \oplus X_3 \\
	X_2             & := X_2 \oplus X_3 \oplus (X_1 << 7) \\
	X_0             & := X_0 <<< 5 \\
	X_2             & := X_2 <<< 22 \\
	\hat B_{i+1}    & := X_0,X_1,X_2,X_3
\end{align*}
\caption{Right part of the figure}
\label{fig:serpentlin}
\end{minipage}
\end{figure}

A more complex figure is shown in Figure~\ref{fig:serpentlin}. \blindtext





\begin{figure}
\centering
\begin{minipage}[b]{0.45\textwidth}
\centering
\begin{tabular}{c}
\begin{lstlisting}
  unsigned char s0[16] = {
          3,  8, 15,  1,
         10,  6,  5, 11,
         14, 13,  4,  2,
          7,  0,  9, 12
  };
\end{lstlisting}
\end{tabular}
\caption{Serpent S-box $S_0$ written as array}
\label{fig:serpents0a}
\end{minipage}
\hspace{0.25cm}
\vline
\hspace{0.25cm}
\begin{minipage}[b]{0.45\textwidth}
\centering
\begin{tabular}{c}
\begin{lstlisting}
#define S0(x0, x1, x2, x3, x4) ({ \
                        x4  = x3; \
  x3 |= x0;  x0 ^= x4;  x4 ^= x2; \
  x4 = ~x4;  x3 ^= x1;  x1 &= x0; \
  x1 ^= x4;  x2 ^= x0;  x0 ^= x3; \
  x4 |= x0;  x0 ^= x2;  x2 &= x1; \
  x3 ^= x2;  x1 = ~x1;  x2 ^= x4; \
  x1 ^= x2;                       \
})
\end{lstlisting}
\end{tabular}
\caption{$S_0$ written as logical sequence}
\label{fig:serpents0l}
\end{minipage}
\end{figure}


A figure using listings is shown in Figure~\ref{fig:serpents0l}. \blindtext

As an example of a complex enumeration, here is
the kernel tree of Linux:
%
\begin{itemize}
	\item {\bf\code{arch/x86}}: x86\_32 and x86\_64 specific source code
	\begin{itemize}
		\item {\bf\code{crypto}}: x86 specific implementation of ciphers
		\item {\bf\code{include/asm}}: x86 specific kernel headers
	\end{itemize}
	\item \code{block}: Block I/O layer
	\item {\bf\code{crypto}}: Crypto API
	\item \code{drivers}: Device drivers
	\item \code{firmware}: Device firmware
	\item \code{fs}: Filesystem implementations
	\item {\bf\code{include}}: Kernel headers
	\begin{itemize}
		\item {\bf\code{crypto}}: Crypto API headers
	\end{itemize}
	\item \code{init}: Kernel boot and initialization code
	\item \code{ipc}: Interprocess communication
	\item \code{kernel}: Core subsystems (e.g. scheduling)
	\item \code{lib}: Helper routines
	\item \code{mm}: Memory Management subsystem
	\item \code{net}: Networking subsystem (Ethernet, IPv4, IPv6, ...)
	\item \code{security}: Linux Security Module
	\item \code{sound}: Sound subsystem
	\item \code{virt}: Virtualization infrastructure
\end{itemize}

\Blindtext[2][1]
%%%%%%%%%%%%%%%%%%%%%%%%%%%%%%%%%%%%%%%%%%%%%%%%%%%%%%%%%%%%%%%%%%%%%%%%%%%%%%%
%
% Implementation
% 
%%%%%%%%%%%%%%%%%%%%%%%%%%%%%%%%%%%%%%%%%%%%%%%%%%%%%%%%%%%%%%%%%%%%%%%%%%%%%%%


\chapter{Implementation}
\label{sec:implementation}


\Blindtext[5][2]

\begin{figure}
\centering
\begin{tabular}{c}
\begin{lstlisting}
static int __init serpent_init(void)
{
    u64 xcr0;
    if (!cpu_has_avx || !cpu_has_osxsave) {
        printk(KERN_INFO "AVX instructions are not detected.\n");
        return -ENODEV;
    }
    xcr0 = xgetbv(XCR_XFEATURE_ENABLED_MASK);
    if ((xcr0 & (XSTATE_SSE | XSTATE_YMM)) != (XSTATE_SSE | XSTATE_YMM)) {
        printk(KERN_INFO "AVX detected but unusable.\n");
        return -ENODEV;
    }
    return crypto_register_algs(serpent_algs, ARRAY_SIZE(serpent_algs));
}

static void __exit serpent_exit(void)
{
    crypto_unregister_algs(serpent_algs, ARRAY_SIZE(serpent_algs));
}

module_init(serpent_init);
module_exit(serpent_exit);

MODULE_DESCRIPTION("Serpent Cipher Algorithm, AVX optimized");
MODULE_LICENSE("GPL");
MODULE_ALIAS("serpent");
\end{lstlisting}
\end{tabular}
\caption{Serpent AVX module initialization}
\label{fig:serpent_init}
\end{figure}

Some complex code is shown in Figure~\ref{fig:serpent_init}. \blindtext



%%%%%%%%%%%%%%%%%%%%%%%%%%%%%%%%%%%%%%%%%%%%%%%%%%%%%%%%%%%%%%%%%%%%%%%%%%%%%%%
%
% Evaluation
% 
%%%%%%%%%%%%%%%%%%%%%%%%%%%%%%%%%%%%%%%%%%%%%%%%%%%%%%%%%%%%%%%%%%%%%%%%%%%%%%%


\chapter{Evaluation}
\label{sec:evaluation}

\Blindtext[5][1]



\fancyhead[RE]{\leftmark}
%%%%%%%%%%%%%%%%%%%%%%%%%%%%%%%%%%%%%%%%%%%%%%%%%%%%%%%%%%%%%%%%%%%%%%%%%%%%%%%
%
% Conclusion and Future Work
% 
%%%%%%%%%%%%%%%%%%%%%%%%%%%%%%%%%%%%%%%%%%%%%%%%%%%%%%%%%%%%%%%%%%%%%%%%%%%%%%%


\chapter{Conclusion and Future Work}
\label{sec:conclusion_and_future_work}

In this chapter we want to draw conclusions about the work, which has been done
during this thesis. \blindtext


\fancyhead[RE]{\rightmark}


% Bibliography
%\bibliographystyle{plainnat}
\bibliography{thesis}

\cleardoublepage

\end{document}
