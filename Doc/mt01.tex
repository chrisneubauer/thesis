%%%%%%%%%%%%%%%%%%%%%%%%%%%%%%%%%%%%%%%%%%%%%%%%%%%%%%%%%%%%%%%%%%%%%%%%%%%%%%%
%
% Introduction
% 
%%%%%%%%%%%%%%%%%%%%%%%%%%%%%%%%%%%%%%%%%%%%%%%%%%%%%%%%%%%%%%%%%%%%%%%%%%%%%%%

\chapter{Introduction}
% 4-5 pages

Optical Character Recognition (OCR) has been the topic of research for many years, even decades. Several workshops, papers and journals have been published, conferences hold on issues in this field. While there are still many open problems (for instance the accurate recognition of arabic texts, symbols, mathematic formulas or handwriting), the knowledge in this area has already led to the development of several highly accurate systems (especially based on English text).
While already a lot of companies use those systems, there is still a majority of others that do not. But, as these systems grow more accurate each year, it is very likely that the need for ocr systems will grow.
As companies are getting connected and globalized, more and more data have to be handled. Modern keywords such as 'Big Data' or 'Data Mining' show, that there is currently a need for solutions to handle those data.
One of those problems is the management of invoices. Companies all over the world have to manage not only the invoices they generate, but also the ones they retrieve, e.g. from their suppliers. While there are already ERP-systems such as SAP ERP 6.0 that are capable of the generation of invoices, especially invoices of other companies are not that easy to process due to the differences between those documents. In addition to that, especially small but also medium-sized companies are mostly not able to afford such systems.
In order to facilitate and accelerate the process of invoice recognition electronic invoice formats have been introduced. If invoices are sent in such a format, a company is recognize the required fields and handle this invoice. Again, this is often the case for big companies, that have defined contracts with their suppliers or customers and have therefore been able to define an electronic invoicing format to automatically process an invoice.
For every other company, there are still problems: Not every invoice is sent in an electronic format, some are still sent as a normal pdf document or even per post.

 
- ocr is nowadays more accurate and can be used in business
- machine learning allows further improvements
- the hardware specifications of standard pcs nowadays allow more complex calculations in lesser time
- electronic invoice as a future standard for companies
Some general information on the context and setting. %\blindtext{}

%\Blindtext[2][2]

\section{Motivation}

-> new format of FERD which is likely to be introduced as a de-factor standard for european countries
-Costs for invoice processing are high
-> Manual processing costs a lot of time, is error-prone, needs employees
- how to deal with lots of invoices?

Specific motivation for the problem at hand. %\blindtext{}

%\blindtext[2]

\section{Task}

The task of this master thesis is to evaluate different electronic invoice formats. An application shall be designed and implemented that processes invoice forms and is capable of storing them in the invoice format that suits the most.
The application should use machine learning in order to improve processing accuracy over time. Nevertheless, errors during the scan process should be handled by the application itself.
The output of this application should be conform with the definition of the electronic invoice format that has been decided in beforehand.

%\blindtext{}


\section{Related Work}

Other relevant academic work and how it differs from this work, for
example %\citet{shannon_diff} and %\citet{blowfish}. Distinguish between
``textual'' citation, as shown in %\citet{shannon_diff}, and
``parenthesis'' citation %\citep{blowfish}.

%\Blindtext[3][1]


\section{Results}

What has been achieved in this work? %\blindtext{}
- open source application
- able to process multiple invoices
- machine learning concept to improve quality over time

%\blindtext[2][1]

\section{Outline}

How is the thesis structured and why? %\blindtext{}
This document is structured in the following way: In the beginning, several electronic invoice formats are presented, explained and compared against each other. Important criteria for the selection of a format are defined and based on those criteria a decision is being made.
After that, we will explain how we want to process a file to a document in the selected electronic invoice format. Chapter 3 will deal with OCR, the available systems at this time as well as a comparison between them and the selection of one of them (including the explanation why this selection has been made).
As the application should learn and improve results over time, we will also deal with machine learning techniques and choose an appropriate one. Chapter 4 will focus on this issue.
From this point on, we have a good understanding about what we want to achieve, with which technologies and methods as well as necessary tools or frameworks for that. Chapter 5 will now discuss several use-cases of the application and show the architectural concept of the application. The following sections will deal with each module and explain it in-depth. The last section will discuss problems that occurred during the implementation.

-> data tests?

In the end, chapter 6 will conclude about this thesis. The resulting application will be explained briefly again. Issues that are still open as well as ideas that could improve the application are listed.

\section{Acknowledgments}

During the implementation of the application and the creation of this document, several people helped me to achieve this presented work. I would like to thank some people in particular:

To Dr. Peter Wilke, who not only supervised my work, but also put thoughts to things that i have not considered before but were crucial for the application.

To Prof. Dr. Oliveira, who supervised my work during my stay in brasil and who gave me good input especially in the field of OCR.

To Prof. Daniel Weingaertner, who managed my stay in brazil, enrolled me in the university and organized all the necessary documents.

To Daniel Stemler whose engagement enabled me to gain access to over thousand invoice documents in order to get a reasonable amount of data to test on.

And to several other friends that helped me or supported me with advices or discussions about technologies or to clarify my understanding regarding a specific approach.