%%%%%%%%%%%%%%%%%%%%%%%%%%%%%%%%%%%%%%%%%%%%%%%%%%%%%%%%%%%%%%%%%%%%%%%%%%%%%%%
%
% Implementation of the application
% 
%%%%%%%%%%%%%%%%%%%%%%%%%%%%%%%%%%%%%%%%%%%%%%%%%%%%%%%%%%%%%%%%%%%%%%%%%%%%%%%

\chapter{Implementation}
\label{Implementation of the application}

The application is structured by different packages. Each of them providing a specific benefit to the program as a whole. Before speaking about those modules, we will talk about the actual requirements of the application. After that, each module will be explained in detail and how it works. After that, the last section will deal with problems and possible solutions.

\section{Requirements definition}

Focus of this application is the possibility to automatically process forms and retrieve the data of the forms. Hence the application should be able to deal with several files and process them without human help. But, since there is a big variety of forms and every company have different structures, retrieving all necessary information can fail. If this happens, a user has to review scanned documents that contain errors. If it doesn't fail, data should be stored without any additional help.

To improve the process of gathering data, a machine learning approach should be implement that facilitates retrieving data and to speed-up form processing over time.

Output of the application should be a storage of the processed forms, appended with electronic invoice information that is valid against the basic- or comfort-level of the ZugfErd-Invoice standard.

\section{Definition of modules}

\section{Architectural concept}

The application will make use of several design patterns. One used pattern on an architectural level is the MVC-pattern. Hence the graphical user interface will be steered by a controller which retrieves data from the database and shows it to the user using javaFX and .fxml-Files. Input and changes the user makes in the view will be transported by the controller to the model which is stored in the database again.

To access the database we will use classes for each business object. The package BO contains classes that represent a table. A data-access-object (DAO) will be used to retrieve data from the database. To do that, this application will also make use of an object-relational mapping framework (Hibernate) which facilitates the conversation between table data and java objects.

\section{Module 1 - OCR}

\section{Module 2 - ANN}

\section{Module 3 - Converter}

\section{Problems during the implementation}